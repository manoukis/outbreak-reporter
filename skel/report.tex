\documentclass[10pt]{article}

%\usepackage[letterpaper, left=0.75in, right=0.75in, top=1in, bottom=1in]{geometry}
\usepackage[letterpaper, margin=0.75in]{geometry}
\usepackage{graphicx}
\usepackage{float}
\usepackage{gensymb}
\usepackage{xspace}
\usepackage{booktabs}
\usepackage{multirow}
\usepackage{makecell}
\usepackage{enumitem}
\usepackage[super]{natbib}
%\usepackage{natbib}
\usepackage{xcolor}
\usepackage{fancyhdr}

\usepackage{titling}
\setlength{\droptitle}{-3em}
\pretitle{\begin{center}\Large}
\predate{}
\postdate{}
\posttitle{\par\end{center}}

\usepackage{authblk}
\renewcommand\Authfont{\normalsize}
\renewcommand\Affilfont{\footnotesize}
\setlength{\affilsep}{.5em}

\usepackage{listings}
\lstset{basicstyle=\footnotesize\ttfamily,breaklines=true}
\lstset{framextopmargin=1pt,frame=bottomline,commentstyle=\color{gray}}


\usepackage{hyperref}
\hypersetup{
    colorlinks,
    linkcolor={red!50!black},
    citecolor={blue!50!black},
    urlcolor={blue!80!black}
}

% header
\pagestyle{fancy}
\rhead{[Internal use only]}
\lhead{\today}
\renewcommand{\headrulewidth}{0pt}

% read in variables/values generated by run_analysis.py
\input{result_variables.texi}

%% Standard form

\title{Draft Medfly lifecycle predictions for 
\VarOutbreakLocation \VarStartDate \\
\Large{[ \VarShortName]}}
\author[1,2]{Travis C. Collier}
\author[1]{Nicholas C. Manoukis}
\affil[1]{Daniel K. Inouye US Pacific Basin Agricultural Research
Center (PBARC), United States Department of Agriculture,
Agricultural Research Service,
Hilo, Hawaii, 96720, USA}
\affil[2]{email: Travis.Collier@ARS.USDA.gov}
%\affil[3]{email: Nicholas.Manoukis@ARS.USDA.gov}
%\date{\today}
\date{}


\begin{document}
\maketitle
\thispagestyle{fancy}

\section*{Introduction and summary results}

We calculated lifecycle timing predictions
and conducted agent-based eradication simulations
for the Mediterranean fruit fly (\textit{Ceratitis capitata}) infestation
in \VarOutbreakLocation with a start date (nominally last fly find) of \VarStartDate.

This report addresses two questions:
\begin{enumerate}[noitemsep,topsep=0pt,parsep=0pt,partopsep=0pt]
\item How do official lifecycle projections compare with 
thermal accumulation lifecycle calculations done for the same day of the year in previous years (`historic').
\item How do thermal accumulation projections compare with 
MEDFOES\cite{manoukis_computer_2014,manoukis_agent-based_2014,10.12688/f1000research.12817.1}
agent-based simulation (ABS) eradication time projections.
\end{enumerate}

\begin{table}[H]
\centering
\begin{tabular}[c]{| c | c |}
\toprule
%Event & Date \\
%\midrule
start date & \VarStartDate \\
official F1 & \VarSSFA \\
official F2 & \VarSSFB \\
official F3 & \VarSSFC \\
%last temp. data & \VarEndDateOfTempData \\
\bottomrule
\end{tabular}
\caption{
\label{givens_table}
Last fly detection and given official projections}
%and date of most recent temperature data used
\end{table}

\begin{table}[H]
\centering
\input{summary_table.texi}
\caption{
\label{summary_table}
Summary of thermal accumulation (DD) and MED-FOES (ABS) projections 
based on temperature datasets starting on the same day-of-year from previous years.
Values are number of days after start date (\VarStartDate),
except `years (N)'.
`\VarShortName' is this outbreak, calculated using temperature data starting on \VarStartDate through \VarEndDateOfTempData.
`ABS 95\% erad' is the point where 95\% of the ABS simulations 
for a starting date (year) have reached extirpation; it is comparable with DD F3.
}
\end{table}

\begin{figure}[H]
\centering
\includegraphics{summary_plot.pdf}
\caption{
\label{summary_plot}
Quarantine duration from start date based on DD F3 and MED-FOES ABS 95\% eradication threshold for previous years (`historic').
Values are calculated using temperature data from \VarStationDescription (\VarStation) starting on \VarStartDOY 
for various years.  DD based values are shown by $\bigcirc$, and ABS values shown by $\diamondsuit$.  
The actual outbreak year (if any data) is shown by filled symbols.
Dashed lines are medians computed from previous years not including actual outbreak year (usable as forward predictions).
%If available, a red line shows the official F3 lifecycle prediction.
}
\end{figure}

\section*{Temperature data}

Hourly temperature data from the nearby high-quality weather station at 
\textbf{\VarStationDescription} (denoted by the callsign \VarStation) is used to generate
projections here.
\textbf{Temperature data from \VarTempStartDate through \VarEndDateOfTempData is used.}
Data were acquired from NOAA's ISD-lite archive,
outliers identified and removed, 
short gaps (\textless \VarLargeGapSize [HH:MM:SS]) filled with simple interpolation, 
and large gaps filled with day-to-day interpolation.\cite{10.12688/f1000research.12817.1}

`Historic' projections are produced by calculating values using historic temperature data 
staring on \VarStartDOY (month-day) for each of the years \VarTempStartYear through \VarTempEndYear.


\section*{Degree day thermal accumulation}

\autoref{dd_spaghetti} and \autoref{dd_hists} show lifecycle projections based on 
degree-day thermal accumulation.
\textbf{Degree day model for Medfly used:
base temperature = \VarDDBaseTempC \degree C (\VarDDBaseTempF \degree F);
per-generation = \VarGenDDc DDc (\VarGenDDf DDf).}

Degree day calculations were done using single-sine method of \citet{ECY:ECY1969503514}
in concordance with recommendations and regulations including 
United States Code of Federal Regulations: Title 7 Subtitle B Chapter III Part 301.32-10 Treatments (2017)\cite{US-301.32-10}
and California Code of Regulations, Title 3, Section 3406\cite{3-CA-ADC-3406}.

\begin{figure}[H]
\centering
\includegraphics{thermal_accumulation.pdf}
\caption{
\label{dd_spaghetti}
Thermal accumulation [DDc] starting on \VarStartDOY 
computed using \VarStation temperature data for multiple years.
Previous years (same day of year) are shown in gray.
The actual outbreak year (start date \VarStartDate), if any, is in red.}
\end{figure}

\begin{figure}[H]
\centering
\includegraphics{DD_previous_years_lifecycles_histograms.pdf}
\caption{
\label{dd_hists}
Histograms of thermal accumulation lifecycle predictions (days past start date) 
computed using temperature data from previous years.
}
\end{figure}

\section*{MED-FOES agent-based eradication simulations}

\autoref{abs_spaghetti} shows the proportions of MED-FOES simulations 
reaching eradication as a function of days after the start date.
MED-FOES version \VarMEDFOESVersion was used 
to perform \VarMFPnR simulations for each starting date (year)
randomly sampling values from the parameter ranges (see \nameref{configuration})
using a Latin-hypercube sampling procedure (LHS).
See \citet{manoukis_computer_2014} and \citet{manoukis_agent-based_2014} %{10.12688/f1000research.12817.1}
for more information about MED-FOES.


\begin{figure}[H]
\centering
\includegraphics{ABS_previous_years_PQL.pdf}
\caption{
\label{abs_spaghetti}
Proportion of MEDFOES simulations achieving eradication vs days after the start date 
for simulations starting on \VarStartDOY.
Results of simulations run on temperature data from previous years are shown in gray.
The actual outbreak year, if any results available, is shown in red.
}
\end{figure}

%\clearpage

\nocite{*}
%\cite{manoukis_computer_2014}
%\cite{manoukis_agent-based_2014}
%\cite{10.12688/f1000research.12817.1}

{\small\bibliographystyle{unsrtnat}
\bibliography{report}}

\clearpage
\section*{Configuration used}\label{configuration}
%\verbatiminput{report.cfg}
\lstinputlisting[language=python]{main.cfg}

%\clearpage
%\section*{MED-FOES configuraton used}
%\verbatiminput{medfoes/mfp.cfg}

\end{document}
